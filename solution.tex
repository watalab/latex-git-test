\chapter{解析手法}
\label{chap:simulation}

\section{はじめに}
陰解法を用いた場合,行列の反転,すなわち連立一次方程式を解く必要がある.
連立一次方程式の解法には直接法と反復法がある.直接法では厳密に解が求めら
れるが,計算量が多い.反復法は反復計算により近似解を求める方法であり,係
数行列の特性に応じて,高速に解を得られることがある.
CFDでは反復法が多く用いられる.行列の特性によって収束の速度に違いが出る
と考えられるので,各手法を適用し,最適なものを選択する.ここでは次の式で
表される$n$次元の連立一次方程式を解くことを考える.
\begin{equation}
 A\bm{x}=\bm{b}
\label{eq:linear}
\end{equation}
$A$は$n\times n$の係数行列,$x$は$n\times1$の解ベクトル,$b$は
$n\times1$のベクトルを表す.

この章では主に文献\cite{test1}を参考にした.
\section{直接法}
連立1次方程式を行列演算によって直接解く,直接法であるガウスの消去法につ
いて述べる.
\subsection{ガウスの消去法}
ガウスの消去法は.係数行列$A$を一度上三角行列に変形し,それから単位行列
$I$を求めることで解を得る.

ガウスの消去法による演算量は$O(n^3)$に比例して増大するため,次元の大きな
連立方程式を解くのは非現実的となる.このためCFDでは反復法が多く用いられ
る.

\section{反復法}
反復法では,初期値として適当な値を与え,真の解に収束する近似解を反復計算により求める.ただし,係数行列
によっては反復計算が収束せず,解を得ることができない場合がある.反復解が
収束するためには係数行列が対角優位であることが必要である.対角優位とは,
行列の対角成分が他の成分に比べて,その絶対値が大きいような行列である.
\begin{equation}
 |a_{ii}|>\sum_{i\neq j}|a_{ij}|, \;\;(i=1,...,n)
\end{equation}
ここで$a_{ij}$は係数行列の要素.


\section{反復法}
反復法では,初期値として適当な値を与え,真の解に収束する近似解を反復計算により求める.ただし,係数行列
によっては反復計算が収束せず,解を得ることができない場合がある.反復解が
収束するためには係数行列が対角優位であることが必要である.対角優位とは,
行列の対角成分が他の成分に比べて,その絶対値が大きいような行列である.
\begin{equation}
 |a_{ii}|>\sum_{i\neq j}|a_{ij}|, \;\;(i=1,...,n)
\end{equation}
ここで$a_{ij}$は係数行列の要素.

